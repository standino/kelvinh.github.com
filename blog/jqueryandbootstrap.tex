% Created 2014-06-24 二 10:17
\documentclass[11pt]{article}
\usepackage[utf8]{inputenc}
\usepackage[T1]{fontenc}
\usepackage{fixltx2e}
\usepackage{graphicx}
\usepackage{longtable}
\usepackage{float}
\usepackage{wrapfig}
\usepackage{rotating}
\usepackage[normalem]{ulem}
\usepackage{amsmath}
\usepackage{textcomp}
\usepackage{marvosym}
\usepackage{wasysym}
\usepackage{amssymb}
\usepackage{hyperref}
\tolerance=1000
\author{Will}
\date{2014-05-30 五}
\title{jquery, bootstrap 和它们的插件们}
\hypersetup{
  pdfkeywords={jquery,bootstrap,js,插件},
  pdfsubject={本文主要介绍jquery,bootstrap和它们的插件的使用},
  pdfcreator={Emacs 24.4.50.2 (Org mode 8.2.7)}}
\begin{document}

\maketitle
\setcounter{tocdepth}{2}
\tableofcontents


\section*{JQuery}
\label{sec-1}
jQuery 是一个 JavaScript 函数库。jQuery 库包含以下特性:
\begin{itemize}
\item HTML 元素选取
\item HTML 元素操作
\item CSS 操作
\item HTML 事件函数
\item JavaScript 特效和动画
\item HTML DOM 遍历和修改
\item AJAX
\item Utilities
\end{itemize}

\subsection*{JQuery UI}
\label{sec-1-1}
\begin{itemize}
\item jQuery UI 是 jQuery 的一部分, 是在jQuery之上的一套专门用于UI交互的类库.
\item 使用jQuery UI可以实现底层交互, 各种高级特效, 动画, 并且可定制主题.
\end{itemize}

\subsection*{基于 JQuery UI 编写插件 - 定义插件}
\label{sec-1-2}
\begin{verbatim}
$.widget("nmk.progressbar", {
    options: {
        value: 0
    },

    _create: function() {
        this.element.addClass("progressbar");
        this._update();
    },

    _setOption: function(key, value) {
        this.options[key] = value;
        this._update();
    },

    _update: function() {
        var progress = this.options.value + "%";
        this.element.text(progress);
        //trigger a event when progressbar is complate
        if(this.options.value == 100){
              this._trigger('complate',null,{value:100});
       }
    },

   _destroy:function(){
      this.element
          .removeClass('progressbar')
          .text("");
     $.Widget.prototype.destroy.call(this);

  }

});
\end{verbatim}
\subsection*{基于 JQuery UI 编写插件 - 使用插件}
\label{sec-1-3}

\begin{verbatim}
$('.loadding').progressbar(
  {value:1,
   complate: function(){
     console.log('complate event has been triggered!');
     //you can return false or true ;
   }
  });
\end{verbatim}

\subsection*{基于 JQuery UI 编写插件 - 代码说明}
\label{sec-1-4}
\begin{itemize}
\item options:在这里面保存的是widget的配置信息,在创建widget的时候需要设置一些配置参数。
\item element:就是widget作用的dom对象。
\item \_create() 这个方法就是创建widget的方法,在页面调用widget的时候,就会执行此方法,来构建widget。Widget的绝大大多数行为和结构都是在这里创建的。
\item \_setOption():此方法提供了options的属性的设置。
\item \_trigger一般用来回调用户传入options的callback。在插件内部调用\_trigger(‘myEvent’)即相当于调用options里面的myEvent这个回调函数。
\item \_destroy():将widget实例从dom对象上移除,在开发widget的时候一般此方法是必须的。就是移除你自己在dom element上添加的样式和行为以及dom结构
\end{itemize}
\subsection*{JQuery UI Portlet}
\label{sec-1-5}

\begin{verbatim}
$('#portlet-demo').portlet('option', 'add', {
  position: {
    x: 0,
    y: 1
  },
  portlet: {
    attrs: {
      id: '$portlet.id',
      itemId: '$portlet.itemId'
    },
    title: '我的任务',

    singleView:{
      enable: function (){
        $('#myWork_portletDiv').hide();
        $('#myWork_pageDiv').show();
      },
      recovery: function (){
        $('#myWork_portletDiv').show();
        $('#myWork_pageDiv').hide();

      }
    },

    content: {
      type: 'ajax',
      url:'#springUrl('/resources/portletTemplate/')myWork.html',
    }
  }});
\end{verbatim}

\section*{Bootstrap}
\label{sec-2}

\begin{itemize}
\item Bootstrap是Twitter推出的一个用于前端开发的开源工具包。它由Twitter的设计师Mark Otto和Jacob Thornton合作开发,是一个CSS/HTML框架。

\item Bootstrap 是一个用于快速开发 Web 应用程序和网站的前端框架。在现代 Web 开发中,有几个几乎所有的 Web 项目中都需要的组件。Bootstrap 为您提供了所有这些基本的模块 - Grid、Typography、Tables、Forms、Buttons 和 Responsiveness。

\item 此外,还有大量其他有用的前端组件,比如 Dropdowns、Navigation、Modals、Typehead、Pagination、Carousal、Breadcrumb、Tab、Thumbnails、Headers 等等。

\item 有了这些,你可以搭建一个 Web 项目,并让它运行地更快速更轻松。
\end{itemize}


\subsection*{Bootstrap弹出效果 - 页面初始化代码}
\label{sec-2-1}
\begin{verbatim}
<div id="myModal" class="modal hide fade" tabindex="-1" role="dialog" aria-labelledby="myModalLabel" aria-hidden="true">
   <div class="modal-header">
       <button type="button" class="close" data-dismiss="modal" aria-hidden="true">×</button>
       <h5 id="myModalLabel">弹出标题</h5>
   </div>
   <div class="modal-body">
          <p>中间显示的内容</p>
   </div>
   <div class="modal-footer">
       <button class="btn" data-dismiss="modal" aria-hidden="true">取消</button>
       <button class="btn btn-primary">确定</button>
   </div>
</div>
\end{verbatim}

\subsection*{Bootstrap弹出效果 - 设置初始化样式}
\label{sec-2-2}

\begin{verbatim}
<script>
   $(function(){
       $('#myModal').modal({
           backdrop:false,
           keyboard:true,
           show:false
       });
   });
</script>
\end{verbatim}

初始化参数介绍:
\begin{itemize}
\item backdrop(默认true):为模态对话框添加一个背景元素。
\item keyboard(默认true):按下esc键时关闭模态对话框。
\item show(默认true):初始化时即显示模态对话框。
\end{itemize}

\subsection*{Bootstrap弹出效果 - 点击事件}
\label{sec-2-3}

\begin{verbatim}
<a href="javascript:void(0);" onclick="showDiv();">弹出</a>

function showDiv(){
   $('#myModal').modal('show');
}
\end{verbatim}


\subsection*{Bootstrap分页}
\label{sec-2-4}
\begin{verbatim}
function listSalesNoticeInPage(container, page) {
    $.ajax({
            url : "/salesNotice/queryByGroup.json?page="+page+"&pageSize="+20+"&date="+new Date(),
            dataType : "json",
            async:false,
            success : function(data) {
                    var result = "";
                    if (data.rows != null) {
                            $.each(data.rows, function() {
                        row += "<td>" + new Date(this.updateTime).format('yyyy-MM-dd hh:mm:ss') + "</td>";
                    }

                    container.empty().append(result);
                    var totalPages = Math.ceil(data.total/20);
                    var pagerOptions = {
                            bootstrapMajorVersion : 3,
                            currentPage : page,
                            numberOfPages : 10,
                            totalPages : totalPages<=0?1:totalPages,
                            onPageClicked : function(e, originalEvent, type, page) {
                                            e.stopImmediatePropagation();
                                    listSalesNoticeInPage(container, page);
                            },
                tooltipTitles: function (type, page, current) {
                  switch (type) {
                  case "first":
                    return "首页";
                  };
                  $('#salesNotice_pager').bootstrapPaginator(pagerOptions);
            }
    });
}
\end{verbatim}
\subsection*{FCBKcomplete}
\label{sec-2-5}
\begin{verbatim}
$("#publishWork_taskAssignerErpLbl").fcbkcomplete({
    json_url: "/ldap/fcsearch",
    addontab: true,
    maxitems: 100,
    input_min_size: 2,
    height: 10,
    cache: false,
    newel: false,
    filter: false,
    complete_text: "输入电子邮件(@符号的前半部分)进行模糊查找..."
});
\end{verbatim}

\section*{参考资料}
\label{sec-3}

\begin{itemize}
\item \url{http://jira.360buy.com/confluence/pages/viewpage.action?pageId=30154029}
\end{itemize}
% Emacs 24.4.50.2 (Org mode 8.2.7)
\end{document}